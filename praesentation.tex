
\documentclass[18pt]{beamer}

% \usepackage[english]{babel}

\usepackage{graphicx}
\usepackage{amsmath}
\usepackage{amssymb}
\usepackage{amsthm}
\usepackage{mathtools}
\usepackage{subcaption}
\usepackage{hyperref}

% Drawing
\usepackage{verbatim}
\usepackage{tikz}
\usepackage{tikz-uml} % UML Class Diagrams
\tikzumlset{fill class=white}
\usetikzlibrary{arrows.meta, positioning}


\usetheme{kit}
\titleimage{titleImage}
\titlelogo{logos}
\institute{Proseminar Anthropomatik: Von der Theorie zur Anwendung}

% Infos über die Präsentation
\title[Scalable Inductive Process Mining]{Scalable Inductive Process Mining}
\subtitle{Mining process trees from large event logs with guarantees}
\author{Maximilian L. Franz}
\date{24. Dezember 2016}

% Folien
\begin{document}

% Title page
\begin{frame}
    \titlepage
\end{frame}

% Table of contents
\begin{frame}
    \frametitle{Agenda}
    \tableofcontents
\end{frame}

\section{Motivation} % (fold)
\label{sec:}
\begin{frame}
    \frametitle{Why mine processes?}    
    \begin{itemize}
	\item Validate exisiting process models
	\pause
	\item Gather new knowledge
\end{itemize}
\pause
\begin{block}{Problem Statement}
\begin{itemize}
\item Mine Process Model from a log of empirical data
\pause
\item Balance between 
	\begin{itemize}
		\item \emph{fitness}
		\item \emph{simplicity}
		\item \emph{generality}
		\item \emph{precision}
	\end{itemize}
\end{itemize}
\end{block}
\end{frame}

\section{Basic Notation} % (fold)
\label{sec:notation_and_basics}
\begin{frame}
    \frametitle{Logs}    
    \begin{block}{Definition (Log)}
    	\begin{itemize}
			\item A log $L$ is a multi-set of traces $\sigma_i$
			\item A trace $\sigma_i$ is a finite sequence of activities in $\mathcal{A}$: $\sigma_i \in \mathcal{A}^*$ 
			\item For simplicity let $\mathcal{A} := \left\{ a, b, c, \dots \right\}$
    	\end{itemize}
    \end{block}
        \begin{block}{Operators}
    	\begin{itemize}
			\item We consider a set of operators $\bigoplus = \left\{ \times, \rightarrow, \wedge, \circlearrowleft \right\}$

			\item They define relation between logs (like regular expression on languages)
			\begin{itemize}
				\item $\times$: Exclusive choice
				\item $\rightarrow$: Sequence
				\item $\wedge$: Parallel
				\item $\circlearrowleft $: Loop
			\end{itemize}
    	\end{itemize}
    \end{block}
\end{frame}

\begin{frame}
\begin{itemize}
	\item Abstract Representation of a process model
	\item Represent Regular Expression 
\end{itemize}
\frametitle{Process Trees}
	\begin{figure}[h!]
	    \centering
	    \begin{tikzpicture}[nodes={draw, circle}, ->, level 2/.style={sibling distance=15mm}, level distance=0.7cm,]
	    
	    \node{$\rightarrow$}
	    child { node{$a$} } 
	    child { node {$\circlearrowleft$} 
	        child { node {$\rightarrow$} 
	            child {node {$\wedge$}
	                child {node {$\times$}
	                    child {node {b}}
	                    child {node {c}}
	                }
	                child {node {d}}
	            }
	            child {node {e}}
	        }
	        child { node {f} }
	    }
	 
	    \end{tikzpicture}
		\caption{Process tree to log $L = \left\{ \langle a, c, d, e, f\rangle, \langle a, c, d, e, f, d, b, e, f\rangle, \dots \right\}$}
		\label{fig:tree}
	\end{figure}
\end{frame}
% section notation_and_basics (end)



\section{Inductive Mining} % (fold)
\label{sec:inductive_mining}

\begin{frame}{~}
	\begin{center}
		\huge{Inductive Mining}
	\end{center}
\end{frame}

\begin{frame}
    \frametitle{Inductive Mining} 
\begin{block}{Idea}
Log $\rightarrow$ Directly-follows-graph (DFG) $\rightarrow$ Cuts $\rightarrow$ Sub-logs
\end{block}
\begin{itemize}
	\item Consider the log $L_2 = \{ \langle a,b,c,d \rangle^3, \langle a,c,b,d \rangle, \langle a,e,d \rangle^2 \}$
\end{itemize}
\begin{figure}[h!]
    \centering
        \begin{tikzpicture}[
            > = stealth, % arrow head style
            shorten > = 1pt, % don't touch arrow head to node
            auto,
            node distance = 1.5cm, % distance between nodes
            semithick % line style
        ]

        \tikzstyle{every state}=[
            draw = black,
            thick,
            fill = white,
            minimum size = 4mm
        ]

        \node[state] (a) {$a$};
        \node[state] (c) [right of=a] {$c$};
        \node[state] (b) [above of=c] {$b$};
        \node[state] (e) [below of=c] {$e$};
        \node[state] (d) [right of=c] {$d$};

        \path[->] (a) edge node {} (b);
        \path[->] (a) edge node {} (c);
        \path[->] (a) edge node {} (e);
        \path[->] (c) edge[bend right] node {} (b);
        \path[->] (b) edge[bend right] node {} (c);
        \path[->] (b) edge node {} (d);
        \path[->] (c) edge node {} (d);
        \path[->] (e) edge node {} (d);
        
    \draw[red, dashed] (0.5, 1) -- (0.5, -1);
    \draw[red, dashed] (2.5, 1) -- (2.5, -1);
    % \draw[green, dashed] (1, -0.7) -- (2, -0.7);
    % \draw[blue, dashed] (1, 0.7) -- (2, 0.7);
    \end{tikzpicture}

    \caption{DFG $ G_L = G(L_2)$ constructed from $L_2$}
    \label{graph:dfg}
\end{figure}
\begin{itemize}
	\item Found: Sequence Cut: $\rightarrow (a, (bce), d)$ 
\end{itemize}
\end{frame}

\begin{frame}
    \frametitle{Inductive Mining} 
\begin{block}{Idea}
Log $\rightarrow$ Directly-follows-graph (DFG) $\rightarrow$ Cuts $\rightarrow$ Sub-logs
\end{block}
\begin{itemize}
	\item Consider the log $L_2 = \{ \langle a,b,c,d \rangle^3, \langle a,c,b,d \rangle, \langle a,e,d \rangle^2 \}$
	\item Reminder: $\bigoplus = \left\{ \times, \rightarrow, \wedge, \circlearrowleft \right\}$
\end{itemize}
\begin{figure}[h!]
    \centering
        \begin{tikzpicture}[
            > = stealth, % arrow head style
            shorten > = 1pt, % don't touch arrow head to node
            auto,
            node distance = 1.5cm, % distance between nodes
            semithick % line style
        ]

        \tikzstyle{every state}=[
            draw = black,
            thick,
            fill = white,
            minimum size = 4mm
        ]

        \node[state, draw=blue] (a) {$a$};
        \node[state] (c) [right of=a] {$c$};
        \node[state] (b) [above of=c] {$b$};
        \node[state] (e) [below of=c] {$e$};
        \node[state, draw=blue] (d) [right of=c] {$d$};

        \path[->] (c) edge[bend right] node {} (b);
        \path[->] (b) edge[bend right] node {} (c);
        
    % \draw[red, dashed] (0.5, 1) -- (0.5, -1);
    % \draw[red, dashed] (2.5, 1) -- (2.5, -1);
    \draw[green, dashed] (1, -0.7) -- (2, -0.7);
    % \draw[blue, dashed] (1, 0.7) -- (2, 0.7);
    \end{tikzpicture}

    \caption{DFG $ G_L = G(L_2)$ constructed from $L_2$}
    \label{graph:dfg}
\end{figure}
\end{frame}

\begin{frame}
    \frametitle{Inductive Mining - Result} 
\begin{block}{Idea}
Log $\rightarrow$ Directly-follows-graph (DFG) $\rightarrow$ Cuts $\rightarrow$ Sub-logs
\end{block}
\begin{itemize}
	\item Consider the log $L_2 = \{ \langle a,b,c,d \rangle^3, \langle a,c,b,d \rangle, \langle a,e,d \rangle^2 \}$
	\item Reminder: $\bigoplus = \left\{ \times, \rightarrow, \wedge, \circlearrowleft \right\}$
	\item Resulting Tree
\end{itemize}

\begin{figure}[h!]
    \centering
    % \usetikzlibrary{graphdrawing.trees}
    \begin{tikzpicture}[nodes={draw, circle}, ->, level 2/.style={sibling distance=15mm}, level distance=1.0cm,]
 
    % Tree structure
    \node{$\rightarrow$}
    child { node{$a$} } 
        child { node {$\times$} 
            child {node {$\wedge$}
                child {node {b}}
                child {node {c}}
            }
            child {node {e}}
        }
    child { node{$d$} } 
 
    \end{tikzpicture}
\caption{ Process tree $Q_L$ mined from $L_2$ with inductive mining}
\label{fig:mined_tree}
\end{figure}

\end{frame}
% section inductive_mining (end)

\section{Demo} % (fold)
\label{sec:demo}

% section demo (end)
\begin{frame}{~}
	\begin{center}
		\huge{DEMO}
	\end{center}
\end{frame}

\begin{frame}{~}
	\begin{center}
		\huge{Thank you for your attention}
	\end{center}
\end{frame}

\end{document}

